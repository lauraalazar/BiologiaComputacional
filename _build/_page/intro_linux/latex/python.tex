%% Generated by Sphinx.
\def\sphinxdocclass{jupyterBook}
\documentclass[letterpaper,10pt,english]{jupyterBook}
\ifdefined\pdfpxdimen
   \let\sphinxpxdimen\pdfpxdimen\else\newdimen\sphinxpxdimen
\fi \sphinxpxdimen=.75bp\relax
\ifdefined\pdfimageresolution
    \pdfimageresolution= \numexpr \dimexpr1in\relax/\sphinxpxdimen\relax
\fi
%% let collapsible pdf bookmarks panel have high depth per default
\PassOptionsToPackage{bookmarksdepth=5}{hyperref}
%% turn off hyperref patch of \index as sphinx.xdy xindy module takes care of
%% suitable \hyperpage mark-up, working around hyperref-xindy incompatibility
\PassOptionsToPackage{hyperindex=false}{hyperref}
%% memoir class requires extra handling
\makeatletter\@ifclassloaded{memoir}
{\ifdefined\memhyperindexfalse\memhyperindexfalse\fi}{}\makeatother

\PassOptionsToPackage{warn}{textcomp}

\catcode`^^^^00a0\active\protected\def^^^^00a0{\leavevmode\nobreak\ }
\usepackage{cmap}
\usepackage{fontspec}
\defaultfontfeatures[\rmfamily,\sffamily,\ttfamily]{}
\usepackage{amsmath,amssymb,amstext}
\usepackage{polyglossia}
\setmainlanguage{english}



\setmainfont{FreeSerif}[
  Extension      = .otf,
  UprightFont    = *,
  ItalicFont     = *Italic,
  BoldFont       = *Bold,
  BoldItalicFont = *BoldItalic
]
\setsansfont{FreeSans}[
  Extension      = .otf,
  UprightFont    = *,
  ItalicFont     = *Oblique,
  BoldFont       = *Bold,
  BoldItalicFont = *BoldOblique,
]
\setmonofont{FreeMono}[
  Extension      = .otf,
  UprightFont    = *,
  ItalicFont     = *Oblique,
  BoldFont       = *Bold,
  BoldItalicFont = *BoldOblique,
]



\usepackage[Bjarne]{fncychap}
\usepackage[,numfigreset=1,mathnumfig]{sphinx}

\fvset{fontsize=\small}
\usepackage{geometry}


% Include hyperref last.
\usepackage{hyperref}
% Fix anchor placement for figures with captions.
\usepackage{hypcap}% it must be loaded after hyperref.
% Set up styles of URL: it should be placed after hyperref.
\urlstyle{same}


\usepackage{sphinxmessages}



        % Start of preamble defined in sphinx-jupyterbook-latex %
         \usepackage[Latin,Greek]{ucharclasses}
        \usepackage{unicode-math}
        % fixing title of the toc
        \addto\captionsenglish{\renewcommand{\contentsname}{Contents}}
        \hypersetup{
            pdfencoding=auto,
            psdextra
        }
        % End of preamble defined in sphinx-jupyterbook-latex %
        

\title{Introducción a Linux}
\date{Jul 25, 2023}
\release{}
\author{The Jupyter Book community}
\newcommand{\sphinxlogo}{\vbox{}}
\renewcommand{\releasename}{}
\makeindex
\begin{document}

\pagestyle{empty}
\sphinxmaketitle
\pagestyle{plain}
\sphinxtableofcontents
\pagestyle{normal}
\phantomsection\label{\detokenize{intro_linux::doc}}


\begin{DUlineblock}{0em}
\item[] \sphinxstylestrong{\Large Qué es Linux}
\end{DUlineblock}

\sphinxAtStartPar
\sphinxhref{https://www.linux.com/what-is-linux/}{Linux} es un sistema operativo muy versátil. Tiene una curva de aprendizaje que requiere cierto nivel de paciencia, pero afortunadamente hay múltiples {[}tutoriales{]}
(\sphinxurl{https://ryanstutorials.net/linuxtutorial/})

\begin{DUlineblock}{0em}
\item[] \sphinxstylestrong{\large Uso de linux}
\end{DUlineblock}
\begin{itemize}
\item {} 
\sphinxAtStartPar
Sólo Linux: Ubuntu

\item {} 
\sphinxAtStartPar
En Windows: \sphinxhref{https://docs.microsoft.com/en-us/windows/wsl/install-win10}{Windows Subsystem for Linux}
o como una \sphinxhref{https://helpdeskgeek.com/linux-tips/how-to-install-ubuntu-in-virtualbox/}{Virtual Machine}

\end{itemize}

\begin{DUlineblock}{0em}
\item[] \sphinxstylestrong{\large Comandos básicos}
\end{DUlineblock}

\sphinxAtStartPar
En linux es usual el uso de la terminal



\sphinxAtStartPar
Este es un ambiente menos gráfico pero mas sencillo para dar instrucciones…sin embargo hay que saber como darlas para que el sistema responda.

\begin{DUlineblock}{0em}
\item[] \sphinxstylestrong{\large ls (listar)}
\end{DUlineblock}

\sphinxAtStartPar
Cuando se entra al sistema, el directorio donde uno se encuentra es el \sphinxstyleemphasis{home}. Para saber que hay dentro de \sphinxstyleemphasis{home}, escribimos

\begin{sphinxuseclass}{cell}\begin{sphinxVerbatimInput}

\begin{sphinxuseclass}{cell_input}
\begin{sphinxVerbatim}[commandchars=\\\{\}]
\PYGZdl{} ls
\end{sphinxVerbatim}

\end{sphinxuseclass}\end{sphinxVerbatimInput}
\begin{sphinxVerbatimOutput}

\begin{sphinxuseclass}{cell_output}
\begin{sphinxVerbatim}[commandchars=\\\{\}]
  Cell In[1], line 1
    \PYGZdl{} ls
    \PYGZca{}
SyntaxError: invalid syntax
\end{sphinxVerbatim}

\end{sphinxuseclass}\end{sphinxVerbatimOutput}

\end{sphinxuseclass}
\sphinxAtStartPar
El comando ls lista el contenido de directorio de trabajo (current directory).

\sphinxAtStartPar
Pero ls no lista realmene todos los archivos, solo aquellos que son visibles. Los archivos que comienzan con un punto (.) y se conocen como ocultos. Usualmente contienen informacion importante de configuracion que no debe ser alterada, a menos que sepamos bien que estamos haciendo.

\sphinxAtStartPar
Ahora listemos todos los archivos del directorio, incluyendo los ocultos:

\begin{sphinxuseclass}{cell}\begin{sphinxVerbatimInput}

\begin{sphinxuseclass}{cell_input}
\begin{sphinxVerbatim}[commandchars=\\\{\}]
\PYGZdl{} ls \PYGZhy{}a
\end{sphinxVerbatim}

\end{sphinxuseclass}\end{sphinxVerbatimInput}

\end{sphinxuseclass}
\sphinxAtStartPar
ls es un ejemplo de un comando con opciones: \sphinxhyphen{}a es un ejemplo de una opción. Las opciones cambianel comportamiento del comando.



\begin{DUlineblock}{0em}
\item[] \sphinxstylestrong{\large Estructura de los directorios}
\end{DUlineblock}

\sphinxAtStartPar
Todos los archivos están  agrupados en la estructura del directorio y esta estructura es jerárquica.



\sphinxAtStartPar
Para saber donde nos encontramos en la estructura del directorio usamos el comando “print working directory”

\begin{sphinxuseclass}{cell}\begin{sphinxVerbatimInput}

\begin{sphinxuseclass}{cell_input}
\begin{sphinxVerbatim}[commandchars=\\\{\}]
\PYGZdl{} pwd
\end{sphinxVerbatim}

\end{sphinxuseclass}\end{sphinxVerbatimInput}

\end{sphinxuseclass}
\begin{DUlineblock}{0em}
\item[] \sphinxstylestrong{\large Crear directorios}
\end{DUlineblock}

\begin{DUlineblock}{0em}
\item[] \sphinxstylestrong{\large mkdir (make directory)}
\end{DUlineblock}

\sphinxAtStartPar
Utilizamos este comando para crear un subdirectorio en el directorio donde estamos

\begin{sphinxuseclass}{cell}\begin{sphinxVerbatimInput}

\begin{sphinxuseclass}{cell_input}
\begin{sphinxVerbatim}[commandchars=\\\{\}]
\PYGZdl{} mkdir biocomp\PYGZus{}clase1
\end{sphinxVerbatim}

\end{sphinxuseclass}\end{sphinxVerbatimInput}

\end{sphinxuseclass}
\sphinxAtStartPar
Para ver el directorio que acabamos de crear escribimos

\begin{sphinxuseclass}{cell}\begin{sphinxVerbatimInput}

\begin{sphinxuseclass}{cell_input}
\begin{sphinxVerbatim}[commandchars=\\\{\}]
\PYGZdl{} ls
\end{sphinxVerbatim}

\end{sphinxuseclass}\end{sphinxVerbatimInput}

\end{sphinxuseclass}


\begin{DUlineblock}{0em}
\item[] \sphinxstylestrong{\large Cambiar directorios}
\end{DUlineblock}

\begin{DUlineblock}{0em}
\item[] \sphinxstylestrong{\large cd (change directory)}
\end{DUlineblock}

\sphinxAtStartPar
Utilizamos este comando para cambiar del directorio donde estamos a otro directorio. Para “entrar” al directorio que acabamos de crear escribirmos

\begin{sphinxuseclass}{cell}\begin{sphinxVerbatimInput}

\begin{sphinxuseclass}{cell_input}
\begin{sphinxVerbatim}[commandchars=\\\{\}]
\PYGZdl{} cd biocomp\PYGZus{}clase1
\end{sphinxVerbatim}

\end{sphinxuseclass}\end{sphinxVerbatimInput}

\end{sphinxuseclass}
\sphinxAtStartPar
Escriba ls para ver qué hay en el directorio (debería estar vacío).
Para volver al directorio parental escribimos

\begin{sphinxuseclass}{cell}\begin{sphinxVerbatimInput}

\begin{sphinxuseclass}{cell_input}
\begin{sphinxVerbatim}[commandchars=\\\{\}]
\PYGZdl{} cd ..
\end{sphinxVerbatim}

\end{sphinxuseclass}\end{sphinxVerbatimInput}

\end{sphinxuseclass}
\sphinxAtStartPar
Si seguimos escribiendo  cd ..   recurrentemente vamos subiendo en la jerarquía de directorios

\begin{DUlineblock}{0em}
\item[] \sphinxstylestrong{\large \textasciitilde{} (home)}
\end{DUlineblock}

\sphinxAtStartPar
Si escribimos  cd   sin puntos llegamos al directorio home. Este se denota también como   \textasciitilde{}  

\begin{sphinxuseclass}{cell}\begin{sphinxVerbatimInput}

\begin{sphinxuseclass}{cell_input}
\begin{sphinxVerbatim}[commandchars=\\\{\}]
\PYG{n}{ls} \PYG{o}{\PYGZti{}}\PYG{o}{/}
\end{sphinxVerbatim}

\end{sphinxuseclass}\end{sphinxVerbatimInput}

\end{sphinxuseclass}
\begin{DUlineblock}{0em}
\item[] \sphinxstylestrong{\large Copiar archivos}
\end{DUlineblock}

\begin{DUlineblock}{0em}
\item[] \sphinxstylestrong{\large cp (copy)}
\end{DUlineblock}

\sphinxAtStartPar
El comando para copiar un archivo arch1 en el directorio actual a un archivo identico pero nombrado arch2 es:

\begin{sphinxuseclass}{cell}\begin{sphinxVerbatimInput}

\begin{sphinxuseclass}{cell_input}
\begin{sphinxVerbatim}[commandchars=\\\{\}]
\PYGZdl{} cp arch1 arch2
\end{sphinxVerbatim}

\end{sphinxuseclass}\end{sphinxVerbatimInput}

\end{sphinxuseclass}
\sphinxAtStartPar
Ahora vamos a tomar un archivo de internet y a usar el comando cp para copiarlo en el directorio que creamos.

\sphinxAtStartPar
Primero, cd al directorio biocomp\_clase1:

\begin{sphinxuseclass}{cell}\begin{sphinxVerbatimInput}

\begin{sphinxuseclass}{cell_input}
\begin{sphinxVerbatim}[commandchars=\\\{\}]
\PYGZdl{} cd \PYGZti{}/biocomp\PYGZus{}clase1
\end{sphinxVerbatim}

\end{sphinxuseclass}\end{sphinxVerbatimInput}

\end{sphinxuseclass}
\sphinxAtStartPar
Luego escribe:

\begin{sphinxuseclass}{cell}\begin{sphinxVerbatimInput}

\begin{sphinxuseclass}{cell_input}
\begin{sphinxVerbatim}[commandchars=\\\{\}]
\PYGZdl{} wget http://www.ee.surrey.ac.uk/Teaching/Unix/science.txt
\end{sphinxVerbatim}

\end{sphinxuseclass}\end{sphinxVerbatimInput}

\end{sphinxuseclass}
\sphinxAtStartPar
Qué crees que hace el comando wget?



\begin{DUlineblock}{0em}
\item[] \sphinxstylestrong{Usando {[}TAB{]}}
\end{DUlineblock}

\sphinxAtStartPar
\sphinxcode{\sphinxupquote{{[}TAB{]}}} es muy útil en la línea de comandos de Linux: completa automaticamente los comandos que se utilizan. Intenta, por ejemplo, \sphinxcode{\sphinxupquote{\$ cd \textasciitilde{}/bioc}} y después \sphinxcode{\sphinxupquote{TAB}}. Esto también ayudará a hacer sugerencias en caso de múltiples opciones.

\begin{DUlineblock}{0em}
\item[] \sphinxstylestrong{\large Moviendo archivos}
\end{DUlineblock}

\sphinxAtStartPar
\sphinxstylestrong{mv (move)}

\sphinxAtStartPar
\sphinxcode{\sphinxupquote{mv arch1 arch2}} mueve o cambia el nombre de \sphinxcode{\sphinxupquote{arch1}} a \sphinxcode{\sphinxupquote{arch2}} y por lo tanto solo queda un

\sphinxAtStartPar
Vamos a mover el archivo \sphinxcode{\sphinxupquote{science.bak}} a al folder \sphinxcode{\sphinxupquote{ejercicios}} creado previamente.

\sphinxAtStartPar
Primero, cambia de directorio a \sphinxcode{\sphinxupquote{biocomp\_clase1}}. Despué escribe:

\begin{sphinxVerbatim}[commandchars=\\\{\}]
\PYGZdl{} mv science.bak ejercicios/
\end{sphinxVerbatim}

\sphinxAtStartPar
escribe \sphinxcode{\sphinxupquote{ls}} y \sphinxcode{\sphinxupquote{ls backup}} para ver si funcionó.

\begin{DUlineblock}{0em}
\item[] \sphinxstylestrong{\large Remover archivos y directorios}
\end{DUlineblock}

\sphinxAtStartPar
\sphinxstylestrong{rm (remove), rmdir (remove directory)}

\sphinxAtStartPar
Para borrar (remover) un archivos, usamos el comando \sphinxcode{\sphinxupquote{rm}} command. Como ejemplo, vamos a crear una copia del archivo \sphinxcode{\sphinxupquote{science.txt}} y borrarlo.

\sphinxAtStartPar
dentro del directorio \sphinxcode{\sphinxupquote{biocomp\_clase1}}, escribe:

\begin{sphinxVerbatim}[commandchars=\\\{\}]
\PYGZdl{} cp science.txt tempfile.txt 
\PYGZdl{} ls
\PYGZdl{} rm tempfile.txt
\PYGZdl{} ls
\end{sphinxVerbatim}

\sphinxAtStartPar
Puedes utilizar el comando \sphinxcode{\sphinxupquote{rmdir}} para remover el directorio (aségurate primero que esté vacío). Intenta remover el directorio \sphinxcode{\sphinxupquote{backup}}. No podrás porque linux no te permitirá remover un directorio que no está vacío. Para borrar un directorio (que no esté vacío) con sus subdirectorios puedes usar \sphinxcode{\sphinxupquote{\sphinxhyphen{}r}} (recursivo):

\begin{sphinxVerbatim}[commandchars=\\\{\}]
\PYGZdl{} rm \PYGZhy{}r /path/directory
\end{sphinxVerbatim}

\sphinxAtStartPar
Este comando pedirá confirmación para algunos archivos que sean importantes. Si estás absolutamente seguro de este paso, puedes añadir el comando \sphinxcode{\sphinxupquote{\sphinxhyphen{}f}} (f de forzar):

\begin{sphinxVerbatim}[commandchars=\\\{\}]
\PYGZdl{} rm \PYGZhy{}rf /path/delete
\end{sphinxVerbatim}



\begin{DUlineblock}{0em}
\item[] \sphinxstylestrong{\large Mostrar los contenidos de un archivo en la pantalla}
\end{DUlineblock}

\sphinxAtStartPar
\sphinxstylestrong{less}

\sphinxAtStartPar
El comando \sphinxcode{\sphinxupquote{less}} muestra los contenidos de un archivo en la pantalla. Escribe:

\begin{sphinxVerbatim}[commandchars=\\\{\}]
\PYGZdl{} less science.txt
\end{sphinxVerbatim}

\sphinxAtStartPar
Presiona la \sphinxcode{\sphinxupquote{{[}barra\sphinxhyphen{}espaciadora{]}}} si quieres continuar hacia abajo, y escribe \sphinxcode{\sphinxupquote{{[}q{]}}} si quieres terminar la lectura.

\sphinxAtStartPar
\sphinxstylestrong{head}

\sphinxAtStartPar
El comando \sphinxcode{\sphinxupquote{head}} muestra las diez primeras lineas de un archivo en la pantalla:

\begin{sphinxVerbatim}[commandchars=\\\{\}]
\PYGZdl{} head science.txt
\end{sphinxVerbatim}

\sphinxAtStartPar
luego escribe:

\begin{sphinxVerbatim}[commandchars=\\\{\}]
\PYGZdl{} head \PYGZhy{}5 science.txt
\end{sphinxVerbatim}

\sphinxAtStartPar
Qué diferencia hizo el \sphinxcode{\sphinxupquote{\sphinxhyphen{}5}} en el comando?

\sphinxAtStartPar
\sphinxstylestrong{tail}

\sphinxAtStartPar
El comando tail muestra las últimas 10 lineas de un archivo:

\begin{sphinxVerbatim}[commandchars=\\\{\}]
\PYGZdl{} tail science.txt
\end{sphinxVerbatim}



\begin{DUlineblock}{0em}
\item[] \sphinxstylestrong{\large Searching the contents of a file}
\end{DUlineblock}

\sphinxAtStartPar
\sphinxstylestrong{Búsequeda Simple usando \sphinxcode{\sphinxupquote{less}}}

\sphinxAtStartPar
Usando \sphinxcode{\sphinxupquote{less}}, se puede buscar dentro de un archivo de texto palabras (un patrón). Por ejemplo, para buscar dentro del archivo \sphinxcode{\sphinxupquote{science.txt}} la palabra “science”, escribimos:

\begin{sphinxVerbatim}[commandchars=\\\{\}]
\PYGZdl{} less science.txt
\end{sphinxVerbatim}

\sphinxAtStartPar
después, aún en \sphinxcode{\sphinxupquote{less}}, escribe slash \sphinxcode{\sphinxupquote{{[}/{]}}} seguido por la palabra a buscar:

\begin{sphinxVerbatim}[commandchars=\\\{\}]
\PYG{o}{/}\PYG{n}{science}
\end{sphinxVerbatim}

\sphinxAtStartPar
Y \sphinxcode{\sphinxupquote{{[}enter{]}}}. Escribe \sphinxcode{\sphinxupquote{{[}n{]}}} para buscar por la siguiente ocurrencia en el texto.

\sphinxAtStartPar
\sphinxstylestrong{grep}

\sphinxAtStartPar
\sphinxcode{\sphinxupquote{grep}} busca en el archivo palabras o patrones dentro de un texto:

\begin{sphinxVerbatim}[commandchars=\\\{\}]
\PYGZdl{} grep science science.txt
\end{sphinxVerbatim}

\sphinxAtStartPar
Como se puede ver, \sphinxcode{\sphinxupquote{grep}} muestra cada línea que contiene la palabra science.
Ahora intente escribir la palabra con mayúscula:

\begin{sphinxVerbatim}[commandchars=\\\{\}]
\PYGZdl{} grep Science science.txt
\end{sphinxVerbatim}

\sphinxAtStartPar
El comando \sphinxcode{\sphinxupquote{grep}} es “case sensitive”, distingue entre S y s.

\sphinxAtStartPar
Para ignorar distinciones entre mayúsculas y minúsculas, utilice la opción \sphinxcode{\sphinxupquote{\sphinxhyphen{}i}}:

\begin{sphinxVerbatim}[commandchars=\\\{\}]
\PYGZdl{} grep \PYGZhy{}i science science.txt
\end{sphinxVerbatim}

\sphinxAtStartPar
Para buscar una frase o patrón, se debe encerrarla entre comillas. Por ejemplo, para buscar spinning top, escriba:

\begin{sphinxVerbatim}[commandchars=\\\{\}]
\PYGZdl{} grep \PYGZhy{}i \PYGZsq{}spinning top\PYGZsq{} science.txt
\end{sphinxVerbatim}

\sphinxAtStartPar
Otras opciones de \sphinxcode{\sphinxupquote{grep}} son:
\begin{itemize}
\item {} 
\sphinxAtStartPar
\sphinxcode{\sphinxupquote{\sphinxhyphen{}v}} muestra las líneas que NO contienen el patrón

\item {} 
\sphinxAtStartPar
\sphinxcode{\sphinxupquote{\sphinxhyphen{}n}} precede cada línea que contiene el patrón con el número de línea

\item {} 
\sphinxAtStartPar
\sphinxcode{\sphinxupquote{\sphinxhyphen{}c}} muestra el número total de líneas que contienen el patrón

\end{itemize}

\sphinxAtStartPar
Intente las diferentes opciones y sus combinaciones. Se pueden utilizar mas de una opción al mismo tiempo! Por ejemplo, el número de líneas sin la palabra science or Science es:

\begin{sphinxVerbatim}[commandchars=\\\{\}]
\PYGZdl{} grep \PYGZhy{}ivc science science.txt
\end{sphinxVerbatim}

\begin{DUlineblock}{0em}
\item[] \sphinxstylestrong{\large Conclusiones: la línea de comando}
\end{DUlineblock}

\sphinxAtStartPar
La línea de comando de linux es muy poderoso. Tan poderoso que la gran mayoría de los servidores y super computadoras del mundo funcionan con los sistemas linux (sin interfase gráfica, i.e. usando solamente la línea de comando).

\sphinxAtStartPar
Aquí navegamos por algunos ejemplos simples, pero esto es solo un “abre\sphinxhyphen{}bocas”. Por ejemplo, el comando \sphinxcode{\sphinxupquote{grep}} es muy importante y útil para buscar patrones, and \sphinxcode{\sphinxupquote{wget}} para bajar archivos. Ambos comandos mucho mas rápidos que si se usaran la interfase gráfica.

\sphinxAtStartPar
Para profundizar sobre linux hay varios tutoriales, por ejemplo \sphinxhref{https://ryanstutorials.net/linuxtutorial/}{tutorial linux} o \sphinxhref{https://swcarpentry.github.io/shell-novice/}{este} recomendado por una compañera del curso

\begin{DUlineblock}{0em}
\item[] \sphinxstylestrong{\large Scripts Ejecutables}
\end{DUlineblock}

\sphinxAtStartPar
Ahora que ya estamos un poco familiarizados con la línea de comando, vamos a utilizar algunas herramientas básicas de programación: al digitar un comando, este da una instrucción para recibir una inofrmación. Este proceso se puede \sphinxstylestrong{automatizar} (por ejemplo mover, renombrar y copiar archivos), escribiendo el comando en un \sphinxstylestrong{script}.

\sphinxAtStartPar
Los scripts son la forma mas simple de escribir un programa. Vamos a escribir un \sphinxstylestrong{script bash} y a ejecutarlo.

\sphinxAtStartPar
Dentro del directorio \sphinxcode{\sphinxupquote{biocomp\_clase1}} vamos a crear un nuevo archivo llamado \sphinxcode{\sphinxupquote{miscript.sh}}. Puedes hacer esto utilizando un  editor de texto de terminal llamado \sphinxcode{\sphinxupquote{nano}}

\begin{sphinxVerbatim}[commandchars=\\\{\}]
\PYGZdl{} nano miscript.sh
\end{sphinxVerbatim}

\sphinxAtStartPar
Este comando inicia el editor de texto. A continuación escriba \sphinxcode{\sphinxupquote{echo Hello World!}}. “Hello World!” es una frase icónica en programación y “echo” es una instrucción que reproduce el texto que sigue. Guarde el archivo usando \sphinxcode{\sphinxupquote{{[}ctrl\sphinxhyphen{}o{]} {[}enter{]}}}  y cierre el programa con \sphinxcode{\sphinxupquote{{[}ctrl\sphinxhyphen{}x{]}}}. En la terminal, ejecute el script con el siguiente comando:

\begin{sphinxVerbatim}[commandchars=\\\{\}]
\PYGZdl{} bash miscript.sh
\end{sphinxVerbatim}

\sphinxAtStartPar
\sphinxstylestrong{Bash} es un “interpretador” que lee el archivo y entiende cómo ejecutar el comando.

\begin{DUlineblock}{0em}
\item[] \sphinxstylestrong{\large Hacer un archivo ejecutable}
\end{DUlineblock}

\sphinxAtStartPar
Liste los archivos en el directorio con la opción \sphinxcode{\sphinxupquote{\sphinxhyphen{}l}} para mas información. Here is how it looks like on MyBinder:

\begin{sphinxVerbatim}[commandchars=\\\{\}]
\PYGZdl{} ls \PYGZhy{}l
total 12
\PYGZhy{}rw\PYGZhy{}r\PYGZhy{}\PYGZhy{}r\PYGZhy{}\PYGZhy{} 1 usuario root   18 Jun 26 10:56 miscript.sh
\end{sphinxVerbatim}

\sphinxAtStartPar
La primera lista de caracteres indica los \sphinxstylestrong{permisos} del archivo. Aquí vemos que el usuario tiene permiso de lectura (\sphinxcode{\sphinxupquote{r}}) y escritura (\sphinxcode{\sphinxupquote{w}}), pero no de ejecución.
Para entender mejor este tema puede consultar este \sphinxhref{https://ryanstutorials.net/linuxtutorial/permissions.php}{tutorial sobre permisos}. Vamos a cambiar el permiso de ejecución

\begin{sphinxVerbatim}[commandchars=\\\{\}]
\PYGZdl{} chmod a+x miscript.sh
\end{sphinxVerbatim}

\sphinxAtStartPar
Ahora todos los usuarios tienen permiso para ejecutar el archivo. Como es un script muy simple y sin consecuencias, no nos preocupa que todos tengan permisos. En otros casos es muy importante tener cuidado. Ahora lo podemos ejecutar:

\begin{sphinxVerbatim}[commandchars=\\\{\}]
\PYGZdl{} ./myscript.sh
\end{sphinxVerbatim}



\begin{sphinxVerbatim}[commandchars=\\\{\}]
\PYGZsh{}!/bin/bash
\PYGZsh{} Script de ejemplo para la práctica 1 de Comp. Biol

echo Hello World!
\end{sphinxVerbatim}

\begin{DUlineblock}{0em}
\item[] \sphinxstylestrong{\large Variables, declaraciones y loops}
\end{DUlineblock}

\sphinxAtStartPar
Una variable es un término que contiene almacenada un dato con información simple y que puede cambiar el dato.

\begin{sphinxuseclass}{cell}\begin{sphinxVerbatimInput}

\begin{sphinxuseclass}{cell_input}
\begin{sphinxVerbatim}[commandchars=\\\{\}]
\PYGZdl{} counter=1
\PYGZdl{} echo \PYGZdl{}counter 
\PYGZdl{} 1
\PYGZdl{} echo counter
\PYGZdl{} counter
\end{sphinxVerbatim}

\end{sphinxuseclass}\end{sphinxVerbatimInput}

\end{sphinxuseclass}
\sphinxAtStartPar
Qué diferencia hay entre \sphinxcode{\sphinxupquote{counter}} y \sphinxcode{\sphinxupquote{\$counter}}? Que la segunda es una \sphinxstylestrong{variable} y por lo tanto puede tomar cualquier valor según lo definamos.

\begin{DUlineblock}{0em}
\item[] \sphinxstylestrong{\large If}
\end{DUlineblock}

\sphinxAtStartPar
Una declaracion (statement) condicional es \sphinxstylestrong{if} y determina que, si un chequeo es verdadero, realice una acción, y si es falso no lo haga. \sphinxstylestrong{If} tiene el siguiente formato:

\sphinxAtStartPar
\sphinxcode{\sphinxupquote{if {[} <un chequeo> {]}}}

\sphinxAtStartPar
\sphinxcode{\sphinxupquote{then}}

\sphinxAtStartPar
\sphinxcode{\sphinxupquote{<comando>}}

\sphinxAtStartPar
\sphinxcode{\sphinxupquote{fi}}

\sphinxAtStartPar
Todo lo que esté entre \sphinxstyleemphasis{then} y \sphinxstyleemphasis{fi} se ejecutará solamente is el chequeo (entre corchetes) es verdadero.
Un ejemplo simple

\begin{sphinxuseclass}{cell}\begin{sphinxVerbatimInput}

\begin{sphinxuseclass}{cell_input}
\begin{sphinxVerbatim}[commandchars=\\\{\}]
\PYGZdl{} num1=10
\PYGZdl{} num2=5

if [ \PYGZdl{}num1 \PYGZhy{}gt \PYGZdl{}num2 ]; then  
echo \PYGZdq{}num1 es \PYGZdl{}num1 y es mayor que \PYGZdl{}num2\PYGZdq{}
fi
\end{sphinxVerbatim}

\end{sphinxuseclass}\end{sphinxVerbatimInput}

\end{sphinxuseclass}
\sphinxAtStartPar
Haga el ejercicio de invertir el chequeo entre corchetes y observe que pasa

\begin{DUlineblock}{0em}
\item[] \sphinxstylestrong{\large While loop}
\end{DUlineblock}

\sphinxAtStartPar
Los loops son muy importantes porque permiten que un comando corra recurrentemente hasta que suceda una situación declarada (statement) o la expresión sea falsa. Este principio es fundamental para la automatización.
El loop \sphinxstylestrong{while} define que, mientras una expresión es verdadera, continúe ejecutando las líneas de código. Tiene el siguiente formato

\sphinxAtStartPar
\sphinxcode{\sphinxupquote{while {[} un chequeo {]}}}

\sphinxAtStartPar
\sphinxcode{\sphinxupquote{do}}

\sphinxAtStartPar
\sphinxcode{\sphinxupquote{<comando>}}

\sphinxAtStartPar
\sphinxcode{\sphinxupquote{done}}

\sphinxAtStartPar
Ahora vamos a escribir un loop para imprimir los numeros del 1 al 10

\begin{sphinxuseclass}{cell}\begin{sphinxVerbatimInput}

\begin{sphinxuseclass}{cell_input}
\begin{sphinxVerbatim}[commandchars=\\\{\}]
\PYGZdl{} counter=1
\PYGZdl{} while [ \PYGZdl{}counter \PYGZhy{}le 10 ]
\PYGZdl{} do
\PYGZdl{} echo \PYGZdl{}counter
\PYGZdl{} ((counter++))
\PYGZdl{} done
\end{sphinxVerbatim}

\end{sphinxuseclass}\end{sphinxVerbatimInput}

\end{sphinxuseclass}
\sphinxAtStartPar
Vamos a descomponer las líneas:
\begin{itemize}
\item {} 
\sphinxAtStartPar
Linea 1: inicializa la variable \sphinxstylestrong{counter} con un valor.

\item {} 
\sphinxAtStartPar
Linea 2: Chequea que mientras (\sphinxstylestrong{While}) la declaración es verdad (counter es menor o igual a 10)

\item {} 
\sphinxAtStartPar
Linea 3: ejecutemos el siguiente comando.

\item {} 
\sphinxAtStartPar
Linea 4: aquí podemos poner cualquier comando. En este caso \sphinxstylestrong{echo} se utiliza porque es simple para ilustrar el propósito.

\item {} 
\sphinxAtStartPar
Linea 6: incrementamos la variable \sphinxstylestrong{counter} por 1 {[}counter++{]}.

\item {} 
\sphinxAtStartPar
Linea 7: al final del loop, vuelve al inicio a hacer el chequeo. Si es verdad, ejecuta el comando, si es falso conitnua a la siguiente linnea.

\end{itemize}

\sphinxAtStartPar
Este programa que ejecutamos en la terminal lo podemos escribir en un script para correrlo las veces que necesitemos y para registrar lo que hicimos

\begin{sphinxuseclass}{cell}\begin{sphinxVerbatimInput}

\begin{sphinxuseclass}{cell_input}
\begin{sphinxVerbatim}[commandchars=\\\{\}]
\PYGZdl{} nano while\PYGZus{}loop.sh

\PYGZsh{}!/bin/bash
\PYGZsh{} Basic while loop
counter=1
while [ \PYGZdl{}counter \PYGZhy{}le 10 ]
do
echo \PYGZdl{}counter
((counter++))
done
\end{sphinxVerbatim}

\end{sphinxuseclass}\end{sphinxVerbatimInput}

\end{sphinxuseclass}
\sphinxAtStartPar
\sphinxcode{\sphinxupquote{{[}ctrl\sphinxhyphen{}o{]} {[}enter{]}}} y \sphinxcode{\sphinxupquote{{[}ctrl\sphinxhyphen{}x{]}}}



\begin{DUlineblock}{0em}
\item[] \sphinxstylestrong{\large For loop}
\end{DUlineblock}

\sphinxAtStartPar
Este loop toma cada item en una lista (en orden, uno después del otro), asigna ese item como valor a la variable \sphinxstylestrong{var}, ejecuta los comandos entre \sphinxstylestrong{do} y \sphinxstylestrong{done} y vuelve al inicio para tomar el siguiente item y repetir recurrentemente.  Tiene la siguiente sintaxis:

\sphinxAtStartPar
\sphinxcode{\sphinxupquote{for var in <list>}}

\sphinxAtStartPar
\sphinxcode{\sphinxupquote{do}}

\sphinxAtStartPar
\sphinxcode{\sphinxupquote{<comando>}}

\sphinxAtStartPar
\sphinxcode{\sphinxupquote{done}}

\sphinxAtStartPar
El siguiente ejemplo ilustra su uso en un directorio que contiene los archivos fasta de secuencias de covid19. Primero vamos a crear un directorio que se llame \sphinxstylestrong{sars\_covid19} y luego vamos a descargar allí las secuencias del:
\begin{itemize}
\item {} 
\sphinxAtStartPar
genoma

\item {} 
\sphinxAtStartPar
proteinas

\item {} 
\sphinxAtStartPar
cds (genes codificadores de proteínas)

\end{itemize}

\begin{sphinxuseclass}{cell}\begin{sphinxVerbatimInput}

\begin{sphinxuseclass}{cell_input}
\begin{sphinxVerbatim}[commandchars=\\\{\}]
\PYGZdl{} mkdir sars\PYGZus{}covid19
\PYGZdl{} curl http://ftp.ensemblgenomes.org/pub/viruses/fasta/sars\PYGZus{}cov\PYGZus{}2/dna/Sars\PYGZus{}cov\PYGZus{}2.ASM985889v3.dna\PYGZus{}rm.toplevel.fa.gz \PYGZhy{}o ./sars\PYGZus{}covid19/Sars\PYGZus{}cov.dna.fa.gz
\PYGZdl{} curl http://ftp.ensemblgenomes.org/pub/viruses/fasta/sars\PYGZus{}cov\PYGZus{}2/pep/Sars\PYGZus{}cov\PYGZus{}2.ASM985889v3.pep.all.fa.gz \PYGZhy{}o ./sars\PYGZus{}covid19/Sars\PYGZus{}cov.prot.fa.gz
\PYGZdl{} curl http://ftp.ensemblgenomes.org/pub/viruses/fasta/sars\PYGZus{}cov\PYGZus{}2/cds/Sars\PYGZus{}cov\PYGZus{}2.ASM985889v3.cds.all.fa.gz \PYGZhy{}o ./sars\PYGZus{}covid19/Sars\PYGZus{}cov.cds.fa.gz
 
\end{sphinxVerbatim}

\end{sphinxuseclass}\end{sphinxVerbatimInput}

\end{sphinxuseclass}
\sphinxAtStartPar
\sphinxstylestrong{curl} (“Client url”) es un comando de línea que permite comunicarse con un servidor. La opción \sphinxcode{\sphinxupquote{\sphinxhyphen{}o}} le asigna un nombre al archivo (en este caso simplificamos el nombre). Donde crees que quedaron los archivos?

\begin{sphinxuseclass}{cell}\begin{sphinxVerbatimInput}

\begin{sphinxuseclass}{cell_input}
\begin{sphinxVerbatim}[commandchars=\\\{\}]
\PYG{n}{cd} \PYG{n}{sars}\PYG{o}{\PYGZhy{}}\PYG{n}{covid19}
\PYG{n}{ls}
\PYG{n}{gunzip} \PYG{o}{*}\PYG{o}{.}\PYG{n}{gz}
\end{sphinxVerbatim}

\end{sphinxuseclass}\end{sphinxVerbatimInput}

\end{sphinxuseclass}
\sphinxAtStartPar
Cuando descargamos los archivos los nombramos con un \sphinxcode{\sphinxupquote{./sars\_covid}} que precede al nombre. Esto indica que lo guarde en el directorio llamado \sphinxcode{\sphinxupquote{sars\_covid}} que esta dentro del directorio donde nos encontramos \sphinxcode{\sphinxupquote{.}}

\begin{sphinxuseclass}{cell}\begin{sphinxVerbatimInput}

\begin{sphinxuseclass}{cell_input}
\begin{sphinxVerbatim}[commandchars=\\\{\}]
for f in \PYGZdl{}(ls);
    do wc \PYGZhy{}l \PYGZdl{}f;
        done
\end{sphinxVerbatim}

\end{sphinxuseclass}\end{sphinxVerbatimInput}

\end{sphinxuseclass}
\sphinxAtStartPar
Note como listamos el directorio usando \sphinxcode{\sphinxupquote{ls}} y lo encerramos en el comanod  \sphinxcode{\sphinxupquote{\$()}}. Esto nos permite crear una variable que contiene el resultado del comando \sphinxcode{\sphinxupquote{ls}}.
Luego iteramos sobre cada item del directorio \sphinxstylestrong{sars\_covid19} y contamos el número de líneas con \sphinxcode{\sphinxupquote{wc \sphinxhyphen{}l}}


\begin{enumerate}
\sphinxsetlistlabels{\arabic}{enumi}{enumii}{}{)}%
\item {} 
\sphinxAtStartPar
Entrar a NCBI \sphinxurl{https://www.ncbi.nlm.nih.gov/}

\item {} 
\sphinxAtStartPar
Escoger en el tipo de base de datos “Protein”

\item {} 
\sphinxAtStartPar
En el campo de búsqueda escribir el identificador “AEO86768”

\item {} 
\sphinxAtStartPar
Escoger el formato “fasta”

\item {} 
\sphinxAtStartPar
Ir a “send to” y escoger “file”

\item {} 
\sphinxAtStartPar
Descargar la secuencia

\item {} 
\sphinxAtStartPar
Crear una carpeta nueva

\item {} 
\sphinxAtStartPar
Renombrar el archivo descargado (debería estar en “\textasciitilde{}/Downloads”) y copiarlo en la nueva carpeta

\item {} 
\sphinxAtStartPar
Repetir el proceso con los siguientes identificadores en NCBI: “AHN64774”, “AAZ67052”, “QIQ54048”

\item {} 
\sphinxAtStartPar
crear un script en bash (con el encabezado sobre el interpretador y un comentario del propósito) que itere sobre los archivos del directorio y despliegue los nombres de los archivos. Nombre el script script\_eval1\_sunombre.sh

\item {} 
\sphinxAtStartPar
ejecute script\_eval1\_sunombre.sh (tenga en cuenta los permisos)

\item {} 
\sphinxAtStartPar
ejecute el comando: cat script\_eval1\_sunombre.sh > reporte1.txt

\item {} 
\sphinxAtStartPar
Ejecute el comando: history >> reporte1.txt

\item {} 
\sphinxAtStartPar
Suba por la asignación de tarea de Teams el archivo reporte.txt

\end{enumerate}









\renewcommand{\indexname}{Index}
\printindex
\end{document}